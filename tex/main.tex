\documentclass[11pt,twoside]{report}
\usepackage{geometry}
\usepackage{enumerate}
\usepackage{latexsym,booktabs}
\usepackage{amsmath,amssymb}
\usepackage{graphicx}
\usepackage[singlespacing]{setspace}
% ===== Defaults^
\usepackage{amsthm}
\usepackage{hyperref}           % hyperlinks
\usepackage{array}              % for \newcolumntype macro
\newcolumntype{C}{>{$}c<{$}}    % math-mode version of "l" column type

\newcommand{\A}{\mathcal{A}} %% I am lazy
\newcommand{\norm}[1]{\left\lVert#1\right\rVert}

\DeclareMathOperator{\MEB}{MEB}
\DeclareMathOperator{\MEBwO}{MEBwO}


\theoremstyle{definition}
% ==== Mine^

\geometry{a4paper,left=2cm,right=2.0cm, top=2cm, bottom=2.0cm}

\newtheorem{definition}{Definition}
\newtheorem{theorem}{Theorem}
\newtheorem{lemma}{Lemma}
\newtheorem{corr}{Corollary}
\newtheorem{prop}{Proposition}
\newtheorem{algorithm}{Algorithm}
\numberwithin{theorem}{section}
\numberwithin{definition}{section}
\numberwithin{lemma}{section}
\numberwithin{algorithm}{section}
\numberwithin{equation}{section}


\begin{document}
\pagestyle{empty}

% =============================================================================
% Title page
% =============================================================================
\begin{titlepage}
\vspace*{.5em}
\center
\textbf{\large{The School of Mathematics}} \\
\vspace*{1em}
\begin{figure}[!h]
\centering
\includegraphics[width=180pt]{CentredLogoCMYK.jpg}
\end{figure}
\vspace{2em}
\textbf{\Huge{Minimum Enclosing Balls with Outliers}}\\[2em]
\textbf{\LARGE{by}}\\
\vspace{2em}
\textbf{\LARGE{Thomas Holmes}}\\
\vspace{6.5em}
\Large{Dissertation Presented for the Degree of\\
MSc in Operational Research with Computational Optimization}\\
\vspace{6.5em}
\Large{August 2021}\\
\vspace{3em}
\Large{Supervised by\\Dr E. Alper Yıldırım}
\vfill
\end{titlepage}

\cleardoublepage

% =============================================================================
% Abstract, acknowledgments, and own work declaration
% =============================================================================
\vspace*{10mm}
\begin{center}
\textbf{\huge{Abstract}}
\end{center}

Here comes your abstract ...

\clearpage
\vspace*{10mm}
\begin{center}
\textbf{\huge{Acknowledgments}}
\end{center}

Here come your acknowledgments ...

\clearpage

\vspace*{10mm}
\begin{center}
\textbf{\huge{Own Work Declaration}}
\end{center}
\vspace*{20mm}

\noindent I declare that this thesis was composed by myself and that the work contained therein is my own, except where explicitly stated otherwise in the text.

\vspace*{10mm}

\begin{flushright}
\textit{(Thomas Holmes)}
\end{flushright}

\cleardoublepage



% =============================================================================
% Table of contents, tables, and pictures (if applicable)
% =============================================================================
\pagestyle{plain}
\setcounter{page}{1}
\pagenumbering{Roman}

\tableofcontents
\clearpage
\listoftables
\listoffigures
\cleardoublepage

\pagenumbering{arabic}
\setcounter{page}{1}

\nocite{*}
\bibliographystyle{abbrv}
\clearpage
% =============================================================================
% Main body
% =============================================================================

% =============================================================================
% Chapter 1
% =============================================================================
\chapter{Introduction}
\section{Motivation}
\section{Literature Review}
\section{Outline}

% =============================================================================
% Chapter 2
% =============================================================================
\chapter{Exact Methods}
\section{Preliminaries}
In this paper we shall denote our data set of finite vectors as $\mathcal{A} = \left\{a^1,\ldots,a^n\right\}\subseteq\mathbb{R}^d$ for $n,d\in\mathbb{N}$.

\begin{definition}
Let $c\in\mathbb{R}^n$ and $r\in\mathbb{R}$. Then the \textit{ball} with center $c$ and radius $r$ is the set
\begin{equation*}
    B(c;r) = \left\{x\in\mathbb{R}^n : \norm{x-c} \leq r\right\}
\end{equation*}
where $\norm{\cdot}:L\to\mathbb{R}$ denotes the standard Euclidean norm on a vector space $L$ (in this paper, $L=\mathbb{R}^n$).
\end{definition}

\begin{definition}
The \textit{minimum enclosing ball} of $\A$, denoted by $\MEB(\A)$, is the ball $B(c^*;r^*)$ where $\A\subseteq B(c^*;r^*)$ and if any $B(c,r)$ exists such that $\mathcal{A}\subseteq B(c,r)$ then $r^*\leq r$.
\end{definition}

\begin{theorem}
For a given set $\A$, $\MEB(\A)$ exists and is unique.
\end{theorem}
\begin{proof}
See \cite[page 5]{two-algorithms}.
\end{proof}

\begin{definition}[{{\cite[page 2]{core-sets}}}]
Let $r^*$ be the radius of $\MEB(\A)$. A ball $B(c;(1+\epsilon)r)$ is a \textit{$(1+\epsilon)$-approximation} of $\MEB(\A)$ if $r\leq r^*$ and $\mathcal{A}\subseteq B(c;(1+\epsilon)r)$.
\end{definition}
\section{Minimum Enclosing Ball}
\begin{definition} \label{MEB}
Let $\mathcal{A} = \left\{a^1,\ldots,a^n\right\}\subseteq\mathbb{R}^d$ for $n,d\in\mathbb{N}$. The optimisation model formulation of the Minimum Enclosing Ball (MEB) problem is as follows:
\begin{center}
    \begin{tabular}{CCC}
        \displaystyle\min_{c,r} & r \\
        \text{s.t.} & \norm{c-a^i} \leq r & i=1,\ldots,n
    \end{tabular}
\end{center}
where $c\in\mathbb{R}^d$ and $r\in\mathbb{R}$ are the decision variables corresponding to the center and radius of the ball respectively.
\end{definition}




\section{Minimum Enclosing Ball with Outliers}
\subsection{Formulation}
\subsection{Analysis of the big M parameter}
\subsection{Benchmarks and Analysis}

% =============================================================================
% Chapter 3
% =============================================================================

\chapter{Algorithmic Methods}
\section{Relaxation-based Heuristic}
\section{Shenmaier Heuristic}
\section{Peeling-based Heuristics}

% =============================================================================
% Chapter 4
% =============================================================================
\chapter{Implementation and Heuristic Benchmarks}
\section{Code Implementation}
\section{Data}
\section{Methodology}
\section{Experiments}
\section{Analysis}

% =============================================================================
% Chapter 5
% =============================================================================
\chapter{Conclusion}

% =============================================================================
%the entries have to be in the file literature.bib
\bibliography{literature}
\clearpage

\appendix
\section*{Appendices}
\addcontentsline{toc}{section}{Appendices}

\section{An Appendix}
\label{app:one}

Some stuff.
\clearpage

\section{Another Appendix}
\label{app:two}

Some other stuff.

\end{document}
